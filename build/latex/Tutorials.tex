%% Generated by Sphinx.
\def\sphinxdocclass{report}
\documentclass[letterpaper,10pt,english]{sphinxmanual}
\ifdefined\pdfpxdimen
   \let\sphinxpxdimen\pdfpxdimen\else\newdimen\sphinxpxdimen
\fi \sphinxpxdimen=.75bp\relax

\PassOptionsToPackage{warn}{textcomp}
\usepackage[utf8]{inputenc}
\ifdefined\DeclareUnicodeCharacter
% support both utf8 and utf8x syntaxes
\edef\sphinxdqmaybe{\ifdefined\DeclareUnicodeCharacterAsOptional\string"\fi}
  \DeclareUnicodeCharacter{\sphinxdqmaybe00A0}{\nobreakspace}
  \DeclareUnicodeCharacter{\sphinxdqmaybe2500}{\sphinxunichar{2500}}
  \DeclareUnicodeCharacter{\sphinxdqmaybe2502}{\sphinxunichar{2502}}
  \DeclareUnicodeCharacter{\sphinxdqmaybe2514}{\sphinxunichar{2514}}
  \DeclareUnicodeCharacter{\sphinxdqmaybe251C}{\sphinxunichar{251C}}
  \DeclareUnicodeCharacter{\sphinxdqmaybe2572}{\textbackslash}
\fi
\usepackage{cmap}
\usepackage[T1]{fontenc}
\usepackage{amsmath,amssymb,amstext}
\usepackage{babel}
\usepackage{times}
\usepackage[Bjarne]{fncychap}
\usepackage{sphinx}

\fvset{fontsize=\small}
\usepackage{geometry}

% Include hyperref last.
\usepackage{hyperref}
% Fix anchor placement for figures with captions.
\usepackage{hypcap}% it must be loaded after hyperref.
% Set up styles of URL: it should be placed after hyperref.
\urlstyle{same}

\addto\captionsenglish{\renewcommand{\figurename}{Fig.\@ }}
\makeatletter
\def\fnum@figure{\figurename\thefigure{}}
\makeatother
\addto\captionsenglish{\renewcommand{\tablename}{Table }}
\makeatletter
\def\fnum@table{\tablename\thetable{}}
\makeatother
\addto\captionsenglish{\renewcommand{\literalblockname}{Listing}}

\addto\captionsenglish{\renewcommand{\literalblockcontinuedname}{continued from previous page}}
\addto\captionsenglish{\renewcommand{\literalblockcontinuesname}{continues on next page}}
\addto\captionsenglish{\renewcommand{\sphinxnonalphabeticalgroupname}{Non-alphabetical}}
\addto\captionsenglish{\renewcommand{\sphinxsymbolsname}{Symbols}}
\addto\captionsenglish{\renewcommand{\sphinxnumbersname}{Numbers}}

\addto\extrasenglish{\def\pageautorefname{page}}

\setcounter{tocdepth}{0}



\title{Tutorials Documentation}
\date{Mar 04, 2021}
\release{}
\author{Lasi Piyathilaka}
\newcommand{\sphinxlogo}{\vbox{}}
\renewcommand{\releasename}{}
\makeindex
\begin{document}

\pagestyle{empty}
\sphinxmaketitle
\pagestyle{plain}
\sphinxtableofcontents
\pagestyle{normal}
\phantomsection\label{\detokenize{index::doc}}



\chapter{ENEX13004:Week 1 - Software Installation}
\label{\detokenize{index:enex13004-week-1-software-installation}}

\section{PC Setup}
\label{\detokenize{_source/week_1/PC_setup:pc-setup}}\label{\detokenize{_source/week_1/PC_setup::doc}}\begin{quote}

In this course we are  utilizing a pre-configured virtual machine.  The second option is to install a native Ubuntu machine with the required software.  The virtual machine approach is by far the easiest option and ensures the fewest build errors during training but is limited in its ability to connect to certain hardware, particularly over USB.
\end{quote}


\subsection{Virtual Machine Configuration (\sphinxstylestrong{Recommended})}
\label{\detokenize{_source/week_1/PC_setup:virtual-machine-configuration-recommended}}
The VM method is the most convenient method of utilizing the training materials: Follow the links below and install Virtual box software and  ROS Melodic training VM
\begin{enumerate}
\def\theenumi{\arabic{enumi}}
\def\labelenumi{\theenumi .}
\makeatletter\def\p@enumii{\p@enumi \theenumi .}\makeatother
\item {} 
\sphinxhref{https://www.virtualbox.org/wiki/Downloads}{Download virtual box}

\item {} 
\sphinxhref{https://cqu365-my.sharepoint.com/:u:/g/personal/l\_piyathilaka\_cqu\_edu\_au/EUAZLBD3DOdHtmZ\_PR-VGZQBm41cAQLCCct00mrjvJGlUQ?e=CjuRzi}{Download ROS Melodic training VM}

\item {} 
\sphinxhref{https://www.virtualbox.org/manual/ch01.html\#ovf}{Import image into virtual box}

\item {} 
Start virtual machine
\begin{enumerate}
\def\theenumii{\arabic{enumii}}
\def\labelenumii{\theenumii .}
\makeatletter\def\p@enumiii{\p@enumii \theenumii .}\makeatother
\item {} 
*Note: If possible, assign two cores in Settings\textgreater{}\textgreater{}System\textgreater{}\textgreater{}Processor to your virtual machine before starting your virtual machine. This setting can be adjusted when the virtual machine is closed and shut down.

\end{enumerate}

\item {} 
Log into virtual machine, user: \sphinxcode{\sphinxupquote{ros-industrial}}, pass: \sphinxcode{\sphinxupquote{rosindustrial}} (no spaces or hyphens)

\end{enumerate}


\chapter{ENEX13004:Week 1 -Ubuntu GUI}
\label{\detokenize{index:enex13004-week-1-ubuntu-gui}}

\section{Navigating the Ubuntu GUI}
\label{\detokenize{_source/week_1/Ubuntu_gui:navigating-the-ubuntu-gui}}\label{\detokenize{_source/week_1/Ubuntu_gui::doc}}\begin{quote}

In this exercise, we will familiarize ourselves with the graphical user interface (GUI) of the Ubuntu operating system.
\end{quote}


\subsection{Task 1: Familiarize Yourself with the Ubuntu Desktop}
\label{\detokenize{_source/week_1/Ubuntu_gui:task-1-familiarize-yourself-with-the-ubuntu-desktop}}
At the log-in screen, click in the password input box, enter \sphinxcode{\sphinxupquote{rosindustrial}} for the password, and hit enter. The screen should look like the image below when you log in:

\sphinxincludegraphics{{ubuntu_desktop}.png}

There are several things you will notice on the desktop:

\sphinxincludegraphics{{ubuntu_desktop_details}.png}
\begin{enumerate}
\def\theenumi{\arabic{enumi}}
\def\labelenumi{\theenumi .}
\makeatletter\def\p@enumii{\p@enumi \theenumi .}\makeatother
\item {} 
The gear icon on the top right of the screen brings up a menu which allows the user to log out, shut down the computer, access system settings, etc…

\item {} 
The bar on the left side shows running and “favorite” applications, connected thumb drives, etc.

\item {} 
The top icon is used to access all applications and files. We will look at this in more detail later.

\item {} 
The next icons are either applications which are currently running or have been “pinned” (again, more on pinning later)

\item {} 
Any removable drives, like thumb drives, are found after the application icons.

\item {} 
If the launcher bar gets “too full”, clicking and dragging up/down allows you to see the applications that are hidden.

\item {} 
To reorganize the icons on the launcher, click and hold the icon until it “pops out”, then move it to the desired location.

\end{enumerate}


\subsection{Task 2: Open and Inspect an Application}
\label{\detokenize{_source/week_1/Ubuntu_gui:task-2-open-and-inspect-an-application}}
Click on the filing-cabinet icon in the launcher. A window should show up, and your desktop should look like something below:

\sphinxincludegraphics{{ubuntu_folder_browser}.png}

Things to notice:
\begin{enumerate}
\def\theenumi{\arabic{enumi}}
\def\labelenumi{\theenumi .}
\makeatletter\def\p@enumii{\p@enumi \theenumi .}\makeatother
\item {} 
The close, minimize, and maximize buttons typically found on the right-hand side of the window title bar are found on the left-hand side.

\item {} 
The menu for windows are found on the menu bar at the top of the screen, much in the same way Macs do. The menus, however, only show up when you hover the mouse over the menu bar.

\item {} 
Notice that there are menu highlights of the folder icon. The dots on the left show how many windows of this application are open. Clicking on these icons when the applications are open does one of two things:

\end{enumerate}
\begin{itemize}
\item {} 
If there is only one window open, this window gets focus.

\item {} 
If more than one are open, clicking a second time causes all of the windows to show up in the foreground, so that you can choose which window to go to (see below):

\end{itemize}

\sphinxincludegraphics{{ubuntu_inspect}.png}


\subsection{Task 3: Start an Application \& Pin it to the Launcher Bar}
\label{\detokenize{_source/week_1/Ubuntu_gui:task-3-start-an-application-pin-it-to-the-launcher-bar}}
Click on the launcher button (top left) and type gedit in the search box. The “Text Editor” application (this is actually gedit) should show up (see below):

\sphinxincludegraphics{{ubuntu_start_application}.png}

Click on the application. The text editor window should show up on the screen, and the text editor icon should show up on the launcher bar on the left-hand side (see below):

\sphinxincludegraphics{{ubuntu_application_pin}.png}
\begin{enumerate}
\def\theenumi{\arabic{enumi}}
\def\labelenumi{\theenumi .}
\makeatletter\def\p@enumii{\p@enumi \theenumi .}\makeatother
\item {} 
Right-click on the text editor launch icon, and select “Lock to Launcher”.

\item {} 
Close the gedit window. The launcher icon should remain after the window closes.

\item {} 
Click on the gedit launcher icon. You should see a new gedit window appear.

\end{enumerate}


\chapter{ENEX13004:Week 1- Linux basics}
\label{\detokenize{index:enex13004-week-1-linux-basics}}

\section{The Linux Terminal}
\label{\detokenize{_source/week_1/linux_basics:the-linux-terminal}}\label{\detokenize{_source/week_1/linux_basics::doc}}\begin{quote}

In this exercise, we will familiarize ourselves with the Linux terminal.
\end{quote}


\subsection{Starting the Terminal}
\label{\detokenize{_source/week_1/linux_basics:starting-the-terminal}}\begin{enumerate}
\def\theenumi{\arabic{enumi}}
\def\labelenumi{\theenumi .}
\makeatletter\def\p@enumii{\p@enumi \theenumi .}\makeatother
\item {} 
To open the terminal, click on the terminal icon:

\sphinxincludegraphics{{ubuntu_terminal_icon}.png}

\item {} 
Create a second terminal window, either by:
\begin{itemize}
\item {} 
Right-clicking on the terminal and selecting the “Open Terminal” or

\item {} 
Selecting “Open Terminal” from the “File” menu

\end{itemize}

\item {} 
Create a second terminal within the same window by pressing “Ctrl+Shift+T” while the terminal window is selected.

\item {} 
Close the 2nd terminal tab, either by:
\begin{itemize}
\item {} 
clicking the small ‘x’ in the terminal tab (not the main terminal window)

\item {} 
typing \sphinxcode{\sphinxupquote{exit}} and hitting enter.

\end{itemize}

\item {} 
The window will have a single line, which looks like this:

\sphinxcode{\sphinxupquote{ros-industrial@ros-i-melodic-vm:\textasciitilde{}\$}}

\item {} 
This is called the prompt, where you enter commands. The prompt, by default, provides three pieces of information:
\begin{enumerate}
\def\theenumii{\arabic{enumii}}
\def\labelenumii{\theenumii .}
\makeatletter\def\p@enumiii{\p@enumii \theenumii .}\makeatother
\item {} 
\sphinxstyleemphasis{ros-industrial} is the login name of the user you are running as.

\item {} 
\sphinxstyleemphasis{ros-i-melodic-vm} is the host name of the computer.

\item {} 
\textasciitilde{} is the directory in which the terminal is currently in. (More on this later).

\end{enumerate}

\item {} 
Close the terminal window by typing \sphinxcode{\sphinxupquote{exit}} or clicking on the red ‘x’ in the window’s titlebar.

\end{enumerate}


\subsection{Navigating Directories and Listing Files}
\label{\detokenize{_source/week_1/linux_basics:navigating-directories-and-listing-files}}

\subsubsection{Home Directory}
\label{\detokenize{_source/week_1/linux_basics:home-directory}}
The directory in which you find yourself when you first login is called your home directory.You will be doing much of your work in your home directory and subdirectories that you’ll be creating to organize your files.
\begin{enumerate}
\def\theenumi{\arabic{enumi}}
\def\labelenumi{\theenumi .}
\makeatletter\def\p@enumii{\p@enumi \theenumi .}\makeatother
\item {} 
You can go in your home directory anytime using the following command \textasciitilde{}.

\end{enumerate}

*\sphinxcode{\sphinxupquote{\$cd \textasciitilde{}}}
\begin{enumerate}
\def\theenumi{\arabic{enumi}}
\def\labelenumi{\theenumi .}
\makeatletter\def\p@enumii{\p@enumi \theenumi .}\makeatother
\item {} 
Here \textasciitilde{} indicates the home directory. Suppose you have to go in any other user’s home directory, use the following command −

\end{enumerate}

\sphinxcode{\sphinxupquote{\$cd \textasciitilde{}username}}
\begin{enumerate}
\def\theenumi{\arabic{enumi}}
\def\labelenumi{\theenumi .}
\makeatletter\def\p@enumii{\p@enumi \theenumi .}\makeatother
\item {} 
To go in your last directory, you can use the following command −

\end{enumerate}

\sphinxcode{\sphinxupquote{\$cd ..}}


\subsubsection{Absolute/Relative Pathnames}
\label{\detokenize{_source/week_1/linux_basics:absolute-relative-pathnames}}
Directories are arranged in a hierarchy with root (/) at the top. The position of any file within the hierarchy is described by its pathname.

Elements of a pathname are separated by a /. A pathname is absolute, if it is described in relation to root, thus absolute pathnames always begin with a /.

Following are some examples of absolute filenames.

\sphinxcode{\sphinxupquote{/etc/passwd /dev/rdsk/Os3}}

A pathname can also be relative to your current working directory. Relative pathnames never begin with /. Relative to user’s home directory, some pathnames might look like this −

\sphinxcode{\sphinxupquote{chem/notes personal/res}}


\subsubsection{ls Command}
\label{\detokenize{_source/week_1/linux_basics:ls-command}}\begin{enumerate}
\def\theenumi{\arabic{enumi}}
\def\labelenumi{\theenumi .}
\makeatletter\def\p@enumii{\p@enumi \theenumi .}\makeatother
\item {} 
Go to the home directory  \sphinxcode{\sphinxupquote{\$cd \textasciitilde{}}} and enter \sphinxcode{\sphinxupquote{ls}} into the terminal.
\begin{itemize}
\item {} 
You should see files like  \sphinxcode{\sphinxupquote{test.txt}}.

\item {} 
Directories, like \sphinxcode{\sphinxupquote{Desktop}}, are colored in blue.

\item {} 
The file \sphinxcode{\sphinxupquote{sample\_job}} is in green; this indicates it has its “execute” bit set, which means it can be executed as a command.

\end{itemize}

\item {} 
Type \sphinxcode{\sphinxupquote{ls *.txt}}.  Only the file \sphinxcode{\sphinxupquote{test.txt}} will be displayed. This will display all the files with .txt extension

\item {} 
Enter \sphinxcode{\sphinxupquote{ls -l}} into the terminal.
\begin{itemize}
\item {} 
Adding the \sphinxcode{\sphinxupquote{-l}} option shows one entry per line, with additional information about each entry in the directory.

\item {} 
The first 10 characters indicate the file type and permissions

\item {} 
The first character is \sphinxcode{\sphinxupquote{d}} if the entry is a directory.

\item {} 
The next 9 characters are the permissions bits for the file

\item {} 
The third and fourth fields are the owning user and group, respectively.

\item {} 
The second-to-last field is the time the file was last modified.

\item {} 
If the file is a symbolic link, the link’s target file is listed after the link’s file name.

\end{itemize}

\item {} 
Enter \sphinxcode{\sphinxupquote{ls -a}} in the terminal.
\begin{itemize}
\item {} 
You will now see  additional files, that starts with “.”. These are hidden files

\end{itemize}

\item {} 
Enter \sphinxcode{\sphinxupquote{ls -a -l}} (or \sphinxcode{\sphinxupquote{ls -al}}) in the command.
\begin{itemize}
\item {} 
You’ll now see all  details of each file such as the creation date and etc.

\end{itemize}

\end{enumerate}


\subsubsection{\sphinxstyleliteralintitle{\sphinxupquote{pwd}} and \sphinxstyleliteralintitle{\sphinxupquote{cd}} Commands}
\label{\detokenize{_source/week_1/linux_basics:pwd-and-cd-commands}}\begin{enumerate}
\def\theenumi{\arabic{enumi}}
\def\labelenumi{\theenumi .}
\makeatletter\def\p@enumii{\p@enumi \theenumi .}\makeatother
\item {} 
Enter \sphinxcode{\sphinxupquote{pwd}} into the terminal.
\begin{itemize}
\item {} 
This will show you the full path of the directory you are working in.To determine where you are within the filesystem hierarchy at any time, enter the command pwd to print the current working directory

\end{itemize}

\item {} 
Enter \sphinxcode{\sphinxupquote{cd new}} into the terminal.
\begin{itemize}
\item {} 
The prompt should change to \sphinxcode{\sphinxupquote{ros-industrial@ros-i-melodic-vm:\textasciitilde{}/new\$}}.

\item {} 
Typing \sphinxcode{\sphinxupquote{pwd}} will show you the path to the current directory the directory \sphinxcode{\sphinxupquote{/home/ros-industrial/new}}.

\end{itemize}

\item {} 
Enter \sphinxcode{\sphinxupquote{cd ..}} into the terminal to go back to the previous directory

\item {} 
Enter \sphinxcode{\sphinxupquote{cd /bin}}, followed by \sphinxcode{\sphinxupquote{ls}}.
\begin{itemize}
\item {} 
This folder contains a list of the most basic Linux commands.
\sphinxstyleemphasis{Note that \sphinxcode{\sphinxupquote{pwd}} and \sphinxcode{\sphinxupquote{ls}} are both in this folder.}

\end{itemize}

\item {} 
Enter \sphinxcode{\sphinxupquote{cd \textasciitilde{}/new}} to return to our working directory.
\begin{itemize}
\item {} 
Linux uses the \sphinxcode{\sphinxupquote{\textasciitilde{}}} character as a shorthand representation for your home directory.

\item {} 
It’s a convenient way to reference files and paths in command-line commands.

\item {} 
You’ll be typing it a lot in this class… remember it!

\end{itemize}

\end{enumerate}

\sphinxstyleemphasis{If you want a full list of options available for any of the commands given in this section, type \sphinxcode{\sphinxupquote{man \textless{}command\textgreater{}}} (where \sphinxcode{\sphinxupquote{\textless{}command\textgreater{}}} is the command you want information on) in the command line.  This will provide you with built-in documentation for the command.  Use the arrow and page up/down keys to scroll, and \sphinxcode{\sphinxupquote{q}} to exit.}


\subsection{Altering Files}
\label{\detokenize{_source/week_1/linux_basics:altering-files}}

\subsubsection{mv Command}
\label{\detokenize{_source/week_1/linux_basics:mv-command}}\begin{enumerate}
\def\theenumi{\arabic{enumi}}
\def\labelenumi{\theenumi .}
\makeatletter\def\p@enumii{\p@enumi \theenumi .}\makeatother
\item {} 
Type \sphinxcode{\sphinxupquote{mv test.txt test2.txt}}, followed by \sphinxcode{\sphinxupquote{ls}}.
\begin{itemize}
\item {} 
You will notice that the file has been renamed to \sphinxcode{\sphinxupquote{test2.txt}}.
\sphinxstyleemphasis{This step shows how \sphinxcode{\sphinxupquote{mv}} can rename files.}

\end{itemize}

\item {} 
Type \sphinxcode{\sphinxupquote{mv test2.txt new}}, then \sphinxcode{\sphinxupquote{ls}}.
\begin{itemize}
\item {} 
The file will no longer be present in the folder.

\end{itemize}

\item {} 
Type \sphinxcode{\sphinxupquote{cd new}}, then \sphinxcode{\sphinxupquote{ls}}.
\begin{itemize}
\item {} 
You will see \sphinxcode{\sphinxupquote{test2.txt}} in the folder.
\sphinxstyleemphasis{These steps show how \sphinxcode{\sphinxupquote{mv}} can move files.}

\end{itemize}

\item {} 
Type \sphinxcode{\sphinxupquote{mv test2.txt ../test.txt}}, then \sphinxcode{\sphinxupquote{ls}}.
\begin{itemize}
\item {} 
\sphinxcode{\sphinxupquote{test2.txt}} will no longer be there.

\end{itemize}

\item {} 
Type \sphinxcode{\sphinxupquote{cd ..}}, then \sphinxcode{\sphinxupquote{ls}}.
\begin{itemize}
\item {} 
You will notice that \sphinxcode{\sphinxupquote{test.txt}} is present again.
\sphinxstyleemphasis{This shows how \sphinxcode{\sphinxupquote{mv}} can move and rename files in one step.}

\end{itemize}

\end{enumerate}


\subsubsection{cp Command}
\label{\detokenize{_source/week_1/linux_basics:cp-command}}\begin{enumerate}
\def\theenumi{\arabic{enumi}}
\def\labelenumi{\theenumi .}
\makeatletter\def\p@enumii{\p@enumi \theenumi .}\makeatother
\item {} 
Type \sphinxcode{\sphinxupquote{cp test.txt new/test2.txt}}, then \sphinxcode{\sphinxupquote{ls new}}.
\begin{itemize}
\item {} 
You will see \sphinxcode{\sphinxupquote{test2.txt}} is now in the \sphinxcode{\sphinxupquote{new}} folder.

\end{itemize}

\item {} 
Type \sphinxcode{\sphinxupquote{cp test.txt "test copy.txt"}}, then \sphinxcode{\sphinxupquote{ls -l}}.
\begin{itemize}
\item {} 
You will see that \sphinxcode{\sphinxupquote{test.txt}} has been copied to \sphinxcode{\sphinxupquote{test copy.txt}}.
\sphinxstyleemphasis{Note that the quotation marks are necessary when spaces or other special characters are included in the file name.}

\end{itemize}

\end{enumerate}


\subsubsection{rm Command}
\label{\detokenize{_source/week_1/linux_basics:rm-command}}\begin{enumerate}
\def\theenumi{\arabic{enumi}}
\def\labelenumi{\theenumi .}
\makeatletter\def\p@enumii{\p@enumi \theenumi .}\makeatother
\item {} 
Type \sphinxcode{\sphinxupquote{rm "test copy.txt"}}, then \sphinxcode{\sphinxupquote{ls -l}}.
\begin{itemize}
\item {} 
You will notice that \sphinxcode{\sphinxupquote{test copy.txt}} is no longer there.

\end{itemize}

\end{enumerate}


\subsubsection{mkdir Command}
\label{\detokenize{_source/week_1/linux_basics:mkdir-command}}\begin{enumerate}
\def\theenumi{\arabic{enumi}}
\def\labelenumi{\theenumi .}
\makeatletter\def\p@enumii{\p@enumi \theenumi .}\makeatother
\item {} 
Type \sphinxcode{\sphinxupquote{mkdir new2}}, then \sphinxcode{\sphinxupquote{ls}}.
\begin{itemize}
\item {} 
You will see there is a new folder \sphinxcode{\sphinxupquote{new2}}.

\end{itemize}

\end{enumerate}


\subsubsection{touch Command}
\label{\detokenize{_source/week_1/linux_basics:touch-command}}\begin{enumerate}
\def\theenumi{\arabic{enumi}}
\def\labelenumi{\theenumi .}
\makeatletter\def\p@enumii{\p@enumi \theenumi .}\makeatother
\item {} 
Type \sphinxcode{\sphinxupquote{touch \textasciitilde{}/Templates/"Untitled Document"}}.
\begin{itemize}
\item {} 
This will create a new Document named \sphinxstylestrong{“Untitled Document”}

\end{itemize}

\end{enumerate}

\sphinxstyleemphasis{You can use the  \sphinxcode{\sphinxupquote{-i}} flag with \sphinxcode{\sphinxupquote{cp}}, \sphinxcode{\sphinxupquote{mv}}, and \sphinxcode{\sphinxupquote{rm}} commands to prompt you when a file will be overwritten or removed.}


\subsection{Job management}
\label{\detokenize{_source/week_1/linux_basics:job-management}}

\subsubsection{Editing Text (and Other GUI Commands)}
\label{\detokenize{_source/week_1/linux_basics:editing-text-and-other-gui-commands}}\begin{enumerate}
\def\theenumi{\arabic{enumi}}
\def\labelenumi{\theenumi .}
\makeatletter\def\p@enumii{\p@enumi \theenumi .}\makeatother
\item {} 
Type \sphinxcode{\sphinxupquote{gedit test.txt}}.
\begin{itemize}
\item {} 
You will notice that a new text editor window will open, and \sphinxcode{\sphinxupquote{test.txt}} will be loaded.

\item {} 
The terminal will not come back with a prompt until the window is closed.

\end{itemize}

\item {} 
There are two ways around this limitation.  Try both…

\item {} 
\sphinxstylestrong{Starting the program and immediately returning a prompt:}
\begin{enumerate}
\def\theenumii{\arabic{enumii}}
\def\labelenumii{\theenumii .}
\makeatletter\def\p@enumiii{\p@enumii \theenumii .}\makeatother
\item {} 
Type \sphinxcode{\sphinxupquote{gedit test.txt \&}}.
\begin{itemize}
\item {} 
The \sphinxcode{\sphinxupquote{\&}} character tells the terminal to run this command in “the background”, meaning the prompt will return immediately.

\end{itemize}

\item {} 
Close the window, then type \sphinxcode{\sphinxupquote{ls}}.
\begin{itemize}
\item {} 
In addition to showing the files, the terminal will notify you that \sphinxcode{\sphinxupquote{gedit}} has finished.

\end{itemize}

\end{enumerate}

\item {} 
\sphinxstylestrong{Moving an already running program into the background:}
\begin{enumerate}
\def\theenumii{\arabic{enumii}}
\def\labelenumii{\theenumii .}
\makeatletter\def\p@enumiii{\p@enumii \theenumii .}\makeatother
\item {} 
Type \sphinxcode{\sphinxupquote{gedit test.txt}}.
\begin{itemize}
\item {} 
The window should open, and the terminal should not have a prompt waiting.

\end{itemize}

\item {} 
In the terminal window, press Ctrl+Z.
\begin{itemize}
\item {} 
The terminal will indicate that \sphinxcode{\sphinxupquote{gedit}} has stopped, and a prompt will appear.

\end{itemize}

\item {} 
Try to use the \sphinxcode{\sphinxupquote{gedit}} window.
\begin{itemize}
\item {} 
Because it is paused, the window will not run.

\end{itemize}

\item {} 
Type \sphinxcode{\sphinxupquote{bg}} in the terminal.
\begin{itemize}
\item {} 
The \sphinxcode{\sphinxupquote{gedit}} window can now run.

\end{itemize}

\item {} 
Close the \sphinxcode{\sphinxupquote{gedit}} window, and type \sphinxcode{\sphinxupquote{ls}} in the terminal window.
\begin{itemize}
\item {} 
As before, the terminal window will indicate that \sphinxcode{\sphinxupquote{gedit}} is finished.

\end{itemize}

\end{enumerate}

\end{enumerate}


\subsubsection{Running Commands as Root}
\label{\detokenize{_source/week_1/linux_basics:running-commands-as-root}}\begin{enumerate}
\def\theenumi{\arabic{enumi}}
\def\labelenumi{\theenumi .}
\makeatletter\def\p@enumii{\p@enumi \theenumi .}\makeatother
\item {} 
In a terminal, type \sphinxcode{\sphinxupquote{ls -a /root}}.
\begin{itemize}
\item {} 
The terminal will indicate that you cannot read the folder \sphinxcode{\sphinxupquote{/root}}.

\item {} 
Many times you will need to run a command that cannot be done as an ordinary user, and must be done as the “super user”

\end{itemize}

\item {} 
To run the previous command as root, add \sphinxcode{\sphinxupquote{sudo}} to the beginning of the command.
\begin{itemize}
\item {} 
In this instance, type \sphinxcode{\sphinxupquote{sudo ls -a /root}} instead.

\item {} 
The terminal will request your password (in this case, \sphinxcode{\sphinxupquote{rosindustrial}}) in order to proceed.

\item {} 
Once you enter the password, you should see the contents of the \sphinxcode{\sphinxupquote{/root}} directory.

\end{itemize}

\end{enumerate}

\sphinxstyleemphasis{\sphinxstylestrong{Warning}: \sphinxcode{\sphinxupquote{sudo}} is a powerful tool which doesn’t provide any sanity checks on what you ask it to do, so be \sphinxstylestrong{VERY} careful in using it.}


\chapter{ENEX13004:Week 1- ROS Node}
\label{\detokenize{index:enex13004-week-1-ros-node}}
\begin{sphinxVerbatim}[commandchars=\\\{\}]
\PYG{n}{Note}\PYG{p}{:} \PYG{n}{This} \PYG{n}{tutorial} \PYG{n}{assumes} \PYG{n}{that} \PYG{n}{you} \PYG{n}{have} \PYG{n}{completed} \PYG{n}{the} \PYG{n}{previous} \PYG{n}{tutorials}\PYG{p}{:} \PYG{n}{building} \PYG{n}{a}
\PYG{n}{ROS} \PYG{n}{package} \PYG{p}{(}\PYG{o}{/}\PYG{n}{ROS}\PYG{o}{/}\PYG{n}{Tutorials}\PYG{o}{/}\PYG{n}{BuildingPackages}\PYG{p}{)}\PYG{o}{.}
\end{sphinxVerbatim}

\begin{sphinxVerbatim}[commandchars=\\\{\}]
\PYG{n}{Please} \PYG{n}{ask} \PYG{n}{about} \PYG{n}{problems} \PYG{o+ow}{and} \PYG{n}{questions} \PYG{n}{regarding} \PYG{n}{this} \PYG{n}{tutorial} \PYG{n}{on} \PYG{n}{answers}\PYG{o}{.}\PYG{n}{ros}\PYG{o}{.}\PYG{n}{org}
\PYG{p}{(}\PYG{n}{http}\PYG{p}{:}\PYG{o}{/}\PYG{o}{/}\PYG{n}{answers}\PYG{o}{.}\PYG{n}{ros}\PYG{o}{.}\PYG{n}{org}\PYG{p}{)}\PYG{o}{.} \PYG{n}{Don}\PYG{l+s+s1}{\PYGZsq{}}\PYG{l+s+s1}{t forget to include in your question the link to this page, the}
\PYG{n}{versions} \PYG{n}{of} \PYG{n}{your} \PYG{n}{OS} \PYG{o}{\PYGZam{}} \PYG{n}{ROS}\PYG{p}{,} \PYG{o+ow}{and} \PYG{n}{also} \PYG{n}{add} \PYG{n}{appropriate} \PYG{n}{tags}\PYG{o}{.}
\end{sphinxVerbatim}


\section{Understanding ROS Nodes}
\label{\detokenize{_source/week_1/ROS_node:understanding-ros-nodes}}\label{\detokenize{_source/week_1/ROS_node::doc}}
\sphinxstylestrong{Description:} This tutorial introduces ROS graph concepts and discusses the use of roscore
(/roscore), rosnode (/rosnode), and rosrun (/rosrun) commandline tools.

\sphinxstylestrong{Tutorial Level:} BEGINNER

\sphinxstylestrong{Next Tutorial:} Understanding ROS topics (/ROS/Tutorials/UnderstandingTopics)

\begin{sphinxVerbatim}[commandchars=\\\{\}]
\PYG{n}{Contents}
\PYG{l+m+mf}{1.} \PYG{n}{Prerequisites}
\PYG{l+m+mf}{2.} \PYG{n}{Quick} \PYG{n}{Overview} \PYG{n}{of} \PYG{n}{Graph} \PYG{n}{Concepts}
\PYG{l+m+mf}{3.} \PYG{n}{Nodes}
\PYG{l+m+mf}{4.} \PYG{n}{Client} \PYG{n}{Libraries}
\PYG{l+m+mf}{5.} \PYG{n}{roscore}
\PYG{l+m+mf}{6.} \PYG{n}{Using} \PYG{n}{rosnode}
\PYG{l+m+mf}{7.} \PYG{n}{Using} \PYG{n}{rosrun}
\PYG{l+m+mf}{8.} \PYG{n}{Review}
\end{sphinxVerbatim}


\subsection{1. Prerequisites}
\label{\detokenize{_source/week_1/ROS_node:prerequisites}}
For this tutorial we’ll use a lighweight simulator, to install it run the following command:

\begin{sphinxVerbatim}[commandchars=\\\{\}]
\PYGZdl{} sudo apt\PYGZhy{}get install ros\PYGZhy{}\PYGZlt{}distro\PYGZgt{}\PYGZhy{}ros\PYGZhy{}tutorials
\end{sphinxVerbatim}

Replace ‘’ with the name of your ROS distribution (e.g. indigo, jade, kinetic)


\subsection{2. Quick Overview of Graph Concepts}
\label{\detokenize{_source/week_1/ROS_node:quick-overview-of-graph-concepts}}
\begin{sphinxVerbatim}[commandchars=\\\{\}]
\PYG{n}{Nodes} \PYG{p}{(}\PYG{o}{/}\PYG{n}{Nodes}\PYG{p}{)}\PYG{p}{:} \PYG{n}{A} \PYG{n}{node} \PYG{o+ow}{is} \PYG{n}{an} \PYG{n}{executable} \PYG{n}{that} \PYG{n}{uses} \PYG{n}{ROS} \PYG{n}{to} \PYG{n}{communicate} \PYG{k}{with} \PYG{n}{other}
\PYG{n}{nodes}\PYG{o}{.}
\PYG{n}{Messages} \PYG{p}{(}\PYG{o}{/}\PYG{n}{Messages}\PYG{p}{)}\PYG{p}{:} \PYG{n}{ROS} \PYG{n}{data} \PYG{n+nb}{type} \PYG{n}{used} \PYG{n}{when} \PYG{n}{subscribing} \PYG{o+ow}{or} \PYG{n}{publishing} \PYG{n}{to} \PYG{n}{a} \PYG{n}{topic}\PYG{o}{.}
\PYG{n}{Topics} \PYG{p}{(}\PYG{o}{/}\PYG{n}{Topics}\PYG{p}{)}\PYG{p}{:} \PYG{n}{Nodes} \PYG{n}{can} \PYG{n}{publish} \PYG{n}{messages} \PYG{n}{to} \PYG{n}{a} \PYG{n}{topic} \PYG{k}{as} \PYG{n}{well} \PYG{k}{as} \PYG{n}{subscribe} \PYG{n}{to} \PYG{n}{a} \PYG{n}{topic} \PYG{n}{to}
\PYG{n}{receive} \PYG{n}{messages}\PYG{o}{.}
\end{sphinxVerbatim}

\begin{sphinxVerbatim}[commandchars=\\\{\}]
\PYG{n}{Master} \PYG{p}{(}\PYG{o}{/}\PYG{n}{Master}\PYG{p}{)}\PYG{p}{:} \PYG{n}{Name} \PYG{n}{service} \PYG{k}{for} \PYG{n}{ROS} \PYG{p}{(}\PYG{n}{i}\PYG{o}{.}\PYG{n}{e}\PYG{o}{.} \PYG{n}{helps} \PYG{n}{nodes} \PYG{n}{find} \PYG{n}{each} \PYG{n}{other}\PYG{p}{)}
\PYG{n}{rosout} \PYG{p}{(}\PYG{o}{/}\PYG{n}{rosout}\PYG{p}{)}\PYG{p}{:} \PYG{n}{ROS} \PYG{n}{equivalent} \PYG{n}{of} \PYG{n}{stdout}\PYG{o}{/}\PYG{n}{stderr}
\PYG{n}{roscore} \PYG{p}{(}\PYG{o}{/}\PYG{n}{roscore}\PYG{p}{)}\PYG{p}{:} \PYG{n}{Master} \PYG{o}{+} \PYG{n}{rosout} \PYG{o}{+} \PYG{n}{parameter} \PYG{n}{server} \PYG{p}{(}\PYG{n}{parameter} \PYG{n}{server} \PYG{n}{will} \PYG{n}{be}
\PYG{n}{introduced} \PYG{n}{later}\PYG{p}{)}
\end{sphinxVerbatim}


\subsection{3. Nodes}
\label{\detokenize{_source/week_1/ROS_node:nodes}}
A node really isn’t much more than an executable file within a ROS package. ROS nodes use a
ROS client library to communicate with other nodes. Nodes can publish or subscribe to a
Topic. Nodes can also provide or use a Service.


\subsection{4. Client Libraries}
\label{\detokenize{_source/week_1/ROS_node:client-libraries}}
ROS client libraries allow nodes written in different programming languages to communicate:

\begin{sphinxVerbatim}[commandchars=\\\{\}]
\PYG{n}{rospy} \PYG{o}{=} \PYG{n}{python} \PYG{n}{client} \PYG{n}{library}
\PYG{n}{roscpp} \PYG{o}{=} \PYG{n}{c}\PYG{o}{+}\PYG{o}{+} \PYG{n}{client} \PYG{n}{library}
\end{sphinxVerbatim}


\subsection{5. roscore}
\label{\detokenize{_source/week_1/ROS_node:roscore}}
roscore is the first thing you should run when using ROS.

Please run:

\begin{sphinxVerbatim}[commandchars=\\\{\}]
\PYGZdl{} roscore
\end{sphinxVerbatim}

You will see something similar to:

\begin{sphinxVerbatim}[commandchars=\\\{\}]
\PYG{o}{.}\PYG{o}{.}\PYG{o}{.} \PYG{n}{logging} \PYG{n}{to} \PYG{o}{\PYGZti{}}\PYG{o}{/}\PYG{o}{.}\PYG{n}{ros}\PYG{o}{/}\PYG{n}{log}\PYG{o}{/}\PYG{l+m+mi}{9}\PYG{n}{cf88ce4}\PYG{o}{\PYGZhy{}}\PYG{n}{b14d}\PYG{o}{\PYGZhy{}}\PYG{l+m+mi}{11}\PYG{n}{df}\PYG{o}{\PYGZhy{}}\PYG{l+m+mi}{8}\PYG{n}{a75}\PYG{o}{\PYGZhy{}}\PYG{l+m+mf}{00251148e8}\PYG{n}{cf}\PYG{o}{/}\PYG{n}{roslaunch}\PYG{o}{\PYGZhy{}}\PYG{n}{mac}
\PYG{n}{hine\PYGZus{}name}\PYG{o}{\PYGZhy{}}\PYG{l+m+mf}{13039.}\PYG{n}{log}
\PYG{n}{Checking} \PYG{n}{log} \PYG{n}{directory} \PYG{k}{for} \PYG{n}{disk} \PYG{n}{usage}\PYG{o}{.} \PYG{n}{This} \PYG{n}{may} \PYG{n}{take} \PYG{n}{awhile}\PYG{o}{.}
\PYG{n}{Press} \PYG{n}{Ctrl}\PYG{o}{\PYGZhy{}}\PYG{n}{C} \PYG{n}{to} \PYG{n}{interrupt}
\PYG{n}{Done} \PYG{n}{checking} \PYG{n}{log} \PYG{n}{file} \PYG{n}{disk} \PYG{n}{usage}\PYG{o}{.} \PYG{n}{Usage} \PYG{o+ow}{is} \PYG{o}{\PYGZlt{}}\PYG{l+m+mi}{1}\PYG{n}{GB}\PYG{o}{.}
\end{sphinxVerbatim}

\begin{sphinxVerbatim}[commandchars=\\\{\}]
\PYG{n}{started} \PYG{n}{roslaunch} \PYG{n}{server} \PYG{n}{http}\PYG{p}{:}\PYG{o}{/}\PYG{o}{/}\PYG{n}{machine\PYGZus{}name}\PYG{p}{:}\PYG{l+m+mi}{33919}\PYG{o}{/}
\PYG{n}{ros\PYGZus{}comm} \PYG{n}{version} \PYG{l+m+mf}{1.4}\PYG{o}{.}
\end{sphinxVerbatim}

\begin{sphinxVerbatim}[commandchars=\\\{\}]
\PYG{n}{SUMMARY}
\PYG{o}{==}\PYG{o}{==}\PYG{o}{==}
\end{sphinxVerbatim}

\begin{sphinxVerbatim}[commandchars=\\\{\}]
\PYG{n}{PARAMETERS}
\PYG{o}{*} \PYG{o}{/}\PYG{n}{rosversion}
\PYG{o}{*} \PYG{o}{/}\PYG{n}{rosdistro}
\end{sphinxVerbatim}

\begin{sphinxVerbatim}[commandchars=\\\{\}]
\PYG{n}{NODES}
\end{sphinxVerbatim}

\begin{sphinxVerbatim}[commandchars=\\\{\}]
\PYG{n}{auto}\PYG{o}{\PYGZhy{}}\PYG{n}{starting} \PYG{n}{new} \PYG{n}{master}
\PYG{n}{process}\PYG{p}{[}\PYG{n}{master}\PYG{p}{]}\PYG{p}{:} \PYG{n}{started} \PYG{k}{with} \PYG{n}{pid} \PYG{p}{[}\PYG{l+m+mi}{13054}\PYG{p}{]}
\PYG{n}{ROS\PYGZus{}MASTER\PYGZus{}URI}\PYG{o}{=}\PYG{n}{http}\PYG{p}{:}\PYG{o}{/}\PYG{o}{/}\PYG{n}{machine\PYGZus{}name}\PYG{p}{:}\PYG{l+m+mi}{11311}\PYG{o}{/}
\end{sphinxVerbatim}

\begin{sphinxVerbatim}[commandchars=\\\{\}]
\PYG{n}{setting} \PYG{o}{/}\PYG{n}{run\PYGZus{}id} \PYG{n}{to} \PYG{l+m+mi}{9}\PYG{n}{cf88ce4}\PYG{o}{\PYGZhy{}}\PYG{n}{b14d}\PYG{o}{\PYGZhy{}}\PYG{l+m+mi}{11}\PYG{n}{df}\PYG{o}{\PYGZhy{}}\PYG{l+m+mi}{8}\PYG{n}{a75}\PYG{o}{\PYGZhy{}}\PYG{l+m+mf}{00251148e8}\PYG{n}{cf}
\PYG{n}{process}\PYG{p}{[}\PYG{n}{rosout}\PYG{o}{\PYGZhy{}}\PYG{l+m+mi}{1}\PYG{p}{]}\PYG{p}{:} \PYG{n}{started} \PYG{k}{with} \PYG{n}{pid} \PYG{p}{[}\PYG{l+m+mi}{13067}\PYG{p}{]}
\PYG{n}{started} \PYG{n}{core} \PYG{n}{service} \PYG{p}{[}\PYG{o}{/}\PYG{n}{rosout}\PYG{p}{]}
\end{sphinxVerbatim}

If roscore does not initialize, you probably have a network configuration issue. See Network
Setup - Single Machine Configuration
(http://www.ros.org/wiki/ROS/NetworkSetup\#Single\_machine\_configuration)

If roscore does not initialize and sends a message about lack of permissions, probably the
\textasciitilde{}/.ros folder is owned by root, change recursively the ownership of that folder with:

\begin{sphinxVerbatim}[commandchars=\\\{\}]
\PYGZdl{} sudo chown \PYGZhy{}R \PYGZlt{}your\PYGZus{}username\PYGZgt{} \PYGZti{}/.ros
\end{sphinxVerbatim}


\subsection{6. Using rosnode}
\label{\detokenize{_source/week_1/ROS_node:using-rosnode}}
Open up a \sphinxstylestrong{new terminal} , and let’s use \sphinxstylestrong{rosnode} to see what running roscore did… Bear in
mind to keep the previous terminal open either by opening a new tab or simply minimizing it.

\begin{sphinxVerbatim}[commandchars=\\\{\}]
\PYG{n}{Note}\PYG{p}{:} \PYG{n}{When} \PYG{n}{opening} \PYG{n}{a} \PYG{n}{new} \PYG{n}{terminal} \PYG{n}{your} \PYG{n}{environment} \PYG{o+ow}{is} \PYG{n}{reset} \PYG{o+ow}{and} \PYG{n}{your} \PYG{o}{\PYGZti{}}\PYG{o}{/}\PYG{o}{.}\PYG{n}{bashrc} \PYG{n}{file} \PYG{o+ow}{is}
\PYG{n}{sourced}\PYG{o}{.} \PYG{n}{If} \PYG{n}{you} \PYG{n}{have} \PYG{n}{trouble} \PYG{n}{running} \PYG{n}{commands} \PYG{n}{like} \PYG{n}{rosnode} \PYG{n}{then} \PYG{n}{you} \PYG{n}{might} \PYG{n}{need} \PYG{n}{to} \PYG{n}{add}
\PYG{n}{some} \PYG{n}{environment} \PYG{n}{setup} \PYG{n}{files} \PYG{n}{to} \PYG{n}{your} \PYG{o}{\PYGZti{}}\PYG{o}{/}\PYG{o}{.}\PYG{n}{bashrc} \PYG{o+ow}{or} \PYG{n}{manually} \PYG{n}{re}\PYG{o}{\PYGZhy{}}\PYG{n}{source} \PYG{n}{them}\PYG{o}{.}
\end{sphinxVerbatim}

rosnode displays information about the ROS nodes that are currently running. The
rosnode list command lists these active nodes:

\begin{sphinxVerbatim}[commandchars=\\\{\}]
\PYGZdl{} rosnode list
\end{sphinxVerbatim}

\begin{sphinxVerbatim}[commandchars=\\\{\}]
\PYG{n}{You} \PYG{n}{will} \PYG{n}{see}\PYG{p}{:}
\end{sphinxVerbatim}

\begin{sphinxVerbatim}[commandchars=\\\{\}]
\PYG{o}{/}\PYG{n}{rosout}
\end{sphinxVerbatim}

This showed us that there is only one node running: rosout (/rosout). This is always running as
it collects and logs nodes’ debugging output.

The rosnode info command returns information about a specific node.

\begin{sphinxVerbatim}[commandchars=\\\{\}]
\PYGZdl{} rosnode info /rosout
\end{sphinxVerbatim}

This gave us some more information about rosout, such as the fact that it publishes
/rosout\_agg.

\begin{sphinxVerbatim}[commandchars=\\\{\}]
\PYG{o}{\PYGZhy{}}\PYG{o}{\PYGZhy{}}\PYG{o}{\PYGZhy{}}\PYG{o}{\PYGZhy{}}\PYG{o}{\PYGZhy{}}\PYG{o}{\PYGZhy{}}\PYG{o}{\PYGZhy{}}\PYG{o}{\PYGZhy{}}\PYG{o}{\PYGZhy{}}\PYG{o}{\PYGZhy{}}\PYG{o}{\PYGZhy{}}\PYG{o}{\PYGZhy{}}\PYG{o}{\PYGZhy{}}\PYG{o}{\PYGZhy{}}\PYG{o}{\PYGZhy{}}\PYG{o}{\PYGZhy{}}\PYG{o}{\PYGZhy{}}\PYG{o}{\PYGZhy{}}\PYG{o}{\PYGZhy{}}\PYG{o}{\PYGZhy{}}\PYG{o}{\PYGZhy{}}\PYG{o}{\PYGZhy{}}\PYG{o}{\PYGZhy{}}\PYG{o}{\PYGZhy{}}\PYG{o}{\PYGZhy{}}\PYG{o}{\PYGZhy{}}\PYG{o}{\PYGZhy{}}\PYG{o}{\PYGZhy{}}\PYG{o}{\PYGZhy{}}\PYG{o}{\PYGZhy{}}\PYG{o}{\PYGZhy{}}\PYG{o}{\PYGZhy{}}\PYG{o}{\PYGZhy{}}\PYG{o}{\PYGZhy{}}\PYG{o}{\PYGZhy{}}\PYG{o}{\PYGZhy{}}\PYG{o}{\PYGZhy{}}\PYG{o}{\PYGZhy{}}\PYG{o}{\PYGZhy{}}\PYG{o}{\PYGZhy{}}\PYG{o}{\PYGZhy{}}\PYG{o}{\PYGZhy{}}\PYG{o}{\PYGZhy{}}\PYG{o}{\PYGZhy{}}\PYG{o}{\PYGZhy{}}\PYG{o}{\PYGZhy{}}\PYG{o}{\PYGZhy{}}\PYG{o}{\PYGZhy{}}\PYG{o}{\PYGZhy{}}\PYG{o}{\PYGZhy{}}\PYG{o}{\PYGZhy{}}\PYG{o}{\PYGZhy{}}\PYG{o}{\PYGZhy{}}\PYG{o}{\PYGZhy{}}\PYG{o}{\PYGZhy{}}\PYG{o}{\PYGZhy{}}\PYG{o}{\PYGZhy{}}\PYG{o}{\PYGZhy{}}\PYG{o}{\PYGZhy{}}\PYG{o}{\PYGZhy{}}\PYG{o}{\PYGZhy{}}\PYG{o}{\PYGZhy{}}\PYG{o}{\PYGZhy{}}\PYG{o}{\PYGZhy{}}\PYG{o}{\PYGZhy{}}\PYG{o}{\PYGZhy{}}\PYG{o}{\PYGZhy{}}\PYG{o}{\PYGZhy{}}\PYG{o}{\PYGZhy{}}\PYG{o}{\PYGZhy{}}\PYG{o}{\PYGZhy{}}\PYG{o}{\PYGZhy{}}
\PYG{n}{Node} \PYG{p}{[}\PYG{o}{/}\PYG{n}{rosout}\PYG{p}{]}
\PYG{n}{Publications}\PYG{p}{:}
\PYG{o}{*} \PYG{o}{/}\PYG{n}{rosout\PYGZus{}agg} \PYG{p}{[}\PYG{n}{rosgraph\PYGZus{}msgs}\PYG{o}{/}\PYG{n}{Log}\PYG{p}{]}
\end{sphinxVerbatim}

\begin{sphinxVerbatim}[commandchars=\\\{\}]
\PYG{n}{Subscriptions}\PYG{p}{:}
\PYG{o}{*} \PYG{o}{/}\PYG{n}{rosout} \PYG{p}{[}\PYG{n}{unknown} \PYG{n+nb}{type}\PYG{p}{]}
\end{sphinxVerbatim}

\begin{sphinxVerbatim}[commandchars=\\\{\}]
\PYG{n}{Services}\PYG{p}{:}
\PYG{o}{*} \PYG{o}{/}\PYG{n}{rosout}\PYG{o}{/}\PYG{n}{get\PYGZus{}loggers}
\PYG{o}{*} \PYG{o}{/}\PYG{n}{rosout}\PYG{o}{/}\PYG{n}{set\PYGZus{}logger\PYGZus{}level}
\end{sphinxVerbatim}

\begin{sphinxVerbatim}[commandchars=\\\{\}]
\PYG{n}{contacting} \PYG{n}{node} \PYG{n}{http}\PYG{p}{:}\PYG{o}{/}\PYG{o}{/}\PYG{n}{machine\PYGZus{}name}\PYG{p}{:}\PYG{l+m+mi}{54614}\PYG{o}{/} \PYG{o}{.}\PYG{o}{.}\PYG{o}{.}
\PYG{n}{Pid}\PYG{p}{:} \PYG{l+m+mi}{5092}
\end{sphinxVerbatim}

Now, let’s see some more nodes. For this, we’re going to use rosrun to bring up another node.


\subsection{7. Using rosrun}
\label{\detokenize{_source/week_1/ROS_node:using-rosrun}}
rosrun allows you to use the package name to directly run a node within a package (without
having to know the package path).

Usage:

\begin{sphinxVerbatim}[commandchars=\\\{\}]
\PYGZdl{} rosrun [package\PYGZus{}name] [node\PYGZus{}name]
\end{sphinxVerbatim}

So now we can run the turtlesim\_node in the turtlesim package.

Then, in a \sphinxstylestrong{new terminal} :

\begin{sphinxVerbatim}[commandchars=\\\{\}]
\PYGZdl{} rosrun turtlesim turtlesim\PYGZus{}node
\end{sphinxVerbatim}

You will see the turtlesim window:

\sphinxstylestrong{NOTE:} The turtle may look different in your turtlesim window. Don’t worry about it - there are
many types of turtle (/Distributions\#Current\_Distribution\_Releases) and yours is a surprise!

In a \sphinxstylestrong{new terminal} :

\begin{sphinxVerbatim}[commandchars=\\\{\}]
\PYGZdl{} rosnode list
\end{sphinxVerbatim}

You will see something similar to:

\begin{sphinxVerbatim}[commandchars=\\\{\}]
\PYG{o}{/}\PYG{n}{rosout}
\PYG{o}{/}\PYG{n}{turtlesim}
\end{sphinxVerbatim}

One powerful feature of ROS is that you can reassign Names from the command-line.

Close the turtlesim window to stop the node (or go back to the rosrun turtlesim terminal and
use ctrl-C). Now let’s re-run it, but this time use a Remapping Argument
(/Remapping\%20Arguments) to change the node’s name:

\begin{sphinxVerbatim}[commandchars=\\\{\}]
\PYGZdl{} rosrun turtlesim turtlesim\PYGZus{}node \PYGZus{}\PYGZus{}name:=my\PYGZus{}turtle
\end{sphinxVerbatim}

Now, if we go back and use rosnode list:

\begin{sphinxVerbatim}[commandchars=\\\{\}]
\PYGZdl{} rosnode list
\end{sphinxVerbatim}

\begin{sphinxVerbatim}[commandchars=\\\{\}]
\PYG{n}{You} \PYG{n}{will} \PYG{n}{see} \PYG{n}{something} \PYG{n}{similar} \PYG{n}{to}\PYG{p}{:}
\end{sphinxVerbatim}

\begin{sphinxVerbatim}[commandchars=\\\{\}]
\PYG{o}{/}\PYG{n}{my\PYGZus{}turtle}
\PYG{o}{/}\PYG{n}{rosout}
\end{sphinxVerbatim}

\begin{sphinxVerbatim}[commandchars=\\\{\}]
Note: If you still see /turtlesim in the list, it might mean that you stopped the node in the
terminal using ctrl\PYGZhy{}C instead of closing the window, or that you don\PYGZsq{}t have the
\PYGZdl{}ROS\PYGZus{}HOSTNAME environment variable defined as described in Network Setup \PYGZhy{} Single
Machine Configuration
(http://www.ros.org/wiki/ROS/NetworkSetup\PYGZsh{}Single\PYGZus{}machine\PYGZus{}configuration). You can try
cleaning the rosnode list with: \PYGZdl{} rosnode cleanup
\end{sphinxVerbatim}

We see our new /my\_turtle node. Let’s use another rosnode command, ping, to test that it’s
up:

\begin{sphinxVerbatim}[commandchars=\\\{\}]
\PYGZdl{} rosnode ping my\PYGZus{}turtle
\end{sphinxVerbatim}

\begin{sphinxVerbatim}[commandchars=\\\{\}]
\PYG{n}{rosnode}\PYG{p}{:} \PYG{n}{node} \PYG{o+ow}{is} \PYG{p}{[}\PYG{o}{/}\PYG{n}{my\PYGZus{}turtle}\PYG{p}{]}
\PYG{n}{pinging} \PYG{o}{/}\PYG{n}{my\PYGZus{}turtle} \PYG{k}{with} \PYG{n}{a} \PYG{n}{timeout} \PYG{n}{of} \PYG{l+m+mf}{3.0}\PYG{n}{s}
\PYG{n}{xmlrpc} \PYG{n}{reply} \PYG{k+kn}{from} \PYG{n+nn}{http}\PYG{p}{:}\PYG{o}{/}\PYG{o}{/}\PYG{n}{aqy}\PYG{p}{:}\PYG{l+m+mi}{42235}\PYG{o}{/} \PYG{n}{time}\PYG{o}{=}\PYG{l+m+mf}{1.152992}\PYG{n}{ms}
\PYG{n}{xmlrpc} \PYG{n}{reply} \PYG{k+kn}{from} \PYG{n+nn}{http}\PYG{p}{:}\PYG{o}{/}\PYG{o}{/}\PYG{n}{aqy}\PYG{p}{:}\PYG{l+m+mi}{42235}\PYG{o}{/} \PYG{n}{time}\PYG{o}{=}\PYG{l+m+mf}{1.120090}\PYG{n}{ms}
\PYG{n}{xmlrpc} \PYG{n}{reply} \PYG{k+kn}{from} \PYG{n+nn}{http}\PYG{p}{:}\PYG{o}{/}\PYG{o}{/}\PYG{n}{aqy}\PYG{p}{:}\PYG{l+m+mi}{42235}\PYG{o}{/} \PYG{n}{time}\PYG{o}{=}\PYG{l+m+mf}{1.700878}\PYG{n}{ms}
\PYG{n}{xmlrpc} \PYG{n}{reply} \PYG{k+kn}{from} \PYG{n+nn}{http}\PYG{p}{:}\PYG{o}{/}\PYG{o}{/}\PYG{n}{aqy}\PYG{p}{:}\PYG{l+m+mi}{42235}\PYG{o}{/} \PYG{n}{time}\PYG{o}{=}\PYG{l+m+mf}{1.127958}\PYG{n}{ms}
\end{sphinxVerbatim}


\subsection{8. Review}
\label{\detokenize{_source/week_1/ROS_node:review}}
What was covered:

\begin{sphinxVerbatim}[commandchars=\\\{\}]
\PYG{n}{roscore} \PYG{o}{=} \PYG{n}{ros}\PYG{o}{+}\PYG{n}{core} \PYG{p}{:} \PYG{n}{master} \PYG{p}{(}\PYG{n}{provides} \PYG{n}{name} \PYG{n}{service} \PYG{k}{for} \PYG{n}{ROS}\PYG{p}{)} \PYG{o}{+} \PYG{n}{rosout} \PYG{p}{(}\PYG{n}{stdout}\PYG{o}{/}\PYG{n}{stderr}\PYG{p}{)} \PYG{o}{+}
\PYG{n}{parameter} \PYG{n}{server} \PYG{p}{(}\PYG{n}{parameter} \PYG{n}{server} \PYG{n}{will} \PYG{n}{be} \PYG{n}{introduced} \PYG{n}{later}\PYG{p}{)}
\PYG{n}{rosnode} \PYG{o}{=} \PYG{n}{ros}\PYG{o}{+}\PYG{n}{node} \PYG{p}{:} \PYG{n}{ROS} \PYG{n}{tool} \PYG{n}{to} \PYG{n}{get} \PYG{n}{information} \PYG{n}{about} \PYG{n}{a} \PYG{n}{node}\PYG{o}{.}
\PYG{n}{rosrun} \PYG{o}{=} \PYG{n}{ros}\PYG{o}{+}\PYG{n}{run} \PYG{p}{:} \PYG{n}{runs} \PYG{n}{a} \PYG{n}{node} \PYG{k+kn}{from} \PYG{n+nn}{a} \PYG{n}{given} \PYG{n}{package}\PYG{o}{.}
\end{sphinxVerbatim}

Now that you understand how ROS nodes work, let’s look at how ROS topics work
(/ROS/Tutorials/UnderstandingTopics). Also, feel free to press Ctrl-C to stop turtlesim\_node.

\begin{sphinxVerbatim}[commandchars=\\\{\}]
\PYG{n}{Wiki}\PYG{p}{:} \PYG{n}{ROS}\PYG{o}{/}\PYG{n}{Tutorials}\PYG{o}{/}\PYG{n}{UnderstandingNodes} \PYG{p}{(}\PYG{n}{last} \PYG{n}{edited} \PYG{l+m+mi}{2019}\PYG{o}{\PYGZhy{}}\PYG{l+m+mi}{06}\PYG{o}{\PYGZhy{}}\PYG{l+m+mi}{05} \PYG{l+m+mi}{23}\PYG{p}{:}\PYG{l+m+mi}{41}\PYG{p}{:}\PYG{l+m+mi}{12} \PYG{n}{by} \PYG{n}{LukeMeier} \PYG{p}{(}\PYG{o}{/}\PYG{n}{LukeMeier}\PYG{p}{)}\PYG{p}{)}
\end{sphinxVerbatim}

Except where
otherwise noted, the
ROS wiki is licensed
under the
Creative Commons Attribution 3.0 (http://creativecommons.org/licenses/by/3.0/)

(https://www.openrobotics.org/)


\chapter{Indices and tables}
\label{\detokenize{index:indices-and-tables}}\begin{itemize}
\item {} 
\DUrole{xref,std,std-ref}{genindex}

\item {} 
\DUrole{xref,std,std-ref}{modindex}

\item {} 
\DUrole{xref,std,std-ref}{search}

\end{itemize}



\renewcommand{\indexname}{Index}
\printindex
\end{document}